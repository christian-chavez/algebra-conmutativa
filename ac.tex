\documentclass[b5paper,10pt]{book}
\usepackage[spanish]{babel}
\usepackage{geometry}
\usepackage{
	xcolor,
	amsmath,
	amsthm,
	amsfonts,
	amssymb,
	enumitem,
}
\geometry{margin=1in}

\newcommand{\red}[1]{\textcolor{red}{#1}}

\theoremstyle{definition}
\newtheorem{thm}{Teorema}
\newtheorem{defi}{Definición}
\newtheorem{problema}{Problema}
\newtheorem{ej}{Ejemplos}

\title{Notas de Álgebra Conmutativa}
\author{Kevin Chávez\\ Université de Sherbrooke}
\date{Compilado: \today}

\begin{document}
\pagestyle{empty}
\maketitle 
\tableofcontents
\pagestyle{plain}

\chapter*{Prefacio}

Estas notas las hago por que no he encontrado una referencia
amigable y en español para estudiantes nuevos en la materia,
y porque es un medio para estudiarla.
No existen aportes originales, todo el material es estándar
y mi única contribución, si es que cabe, es la organización 
del material.
Además, presento solución de todos los problemas propuestos 
al final de cada sección.

El libro de Atiyah y MacDonald es la referencia estándar.
También sugiero bastante el libro de Assem y Coelho.

\chapter{Básico de álgebra conmutativa}

\section{Preliminares}

\subsection{Intuición}

Estudiamos teoría de módulos no por el hecho de estudiar la teoría en sí misma,
sino que la usamos para entender a mayor profundidad la teoría de anillos.
\red{poner la analogía con la teoría de grupos}
De hecho, los módulos que deben captar nuestro mayor interés 
son aquellos que están muy relacionados con el anillo de base,
por ejemplo los ideales, los cocientes y las extensiones de anillos,
demás de los módulos que podamos construir a partir de ellos, como las sumas o productos.

Los módulos mejor comportados son los módulos libres
y los módulos noetherianos.
Practicamente todos los anillos usados en el álgebra conmutativa son
noetherianos.

\subsection{Convenciones}

El símbolo \(\subset\) denota inclusión y no inclusión propia,
así que lo usamos en lugar de \(\subseteq\).
Cuando querramos indicar inclusión propia usaremos \(\subsetneq\).	
Esta convención es uno de los buenos modales
que comparten muchos autores de excelentes libros.

\section{Anillos e ideales}



\begin{defi}
Un anillo es un grupo abeliano \((A,+)\) con una operación binaria
\(\cdot : A\times A\to A\) tal que
\[
	(a,b)\mapsto a\cdot b
\]
que satisface los siguientes axiomas
\end{defi}

El hecho de que nos enfocaremos únicamente
en anillos conmutativos le hace honor al nombre de 
esta disciplina.

Como es de costumbre, al decir 
``A es un anillo'', 
queremos decir que 
\((A,+,\cdot)\) es un anillo
para algunas operaciones de suma \(+\) y multiplicación \(\cdot\).

Contrario a lo que uno podría pensar,
las subestructuras de interés contenidas en \(A\)
no son los subanillos sino sus ideales.

\begin{defi}
Sea \(A\) un anillo.
Un ideal \(I\) en \(A\) es un subgrupo abeliano de \((A,+)\)
que es estable bajo la multiplicación por elementos de \(A\)
\end{defi}
Es decir, un ideal \(I\) es un subconjunto no vacío de \(A\) 
que sarisface
\[
x-y\in I\quad\text{y}\quad a I\subset I, \text{ para todo } x,y\in I, a \in A
\]
La primera condición (junto con el hecho de que \(I\neq \varnothing\))
expresa el hecho de que \(I\) es un subgrupo de \(A\) respecto a la suma,
y la segunda que 
\[
ax \in I \text{ para todo } a\in A \text{ y todo } x\in I.
\]
El ideal cero \(\{0\}\) se denota por \(0\)
y apelaremos al contexto cuando sea necesario
distinguir entre el elemento \(0\) del ideal cero.
Es simplemente un ahorro de notación.

Ideal generado por un conjunto.
Nótese que 
\[
(\varnothing) = 0\quad \text{y}\quad (1) = A.
\]
Ejemplo.
Sea \(I = (a,b)\) es un ideal en un anillo \(A\).
Entonces \[I^2 = II = (a,b)(a,b)= (a^2, ab, b^2)\]
\[
I^n = ? 
\]
Si \(I = (a,b)\) y \(J=(c,d)\),
\[
IJ = (ac,ad,bc,bd).
\]
Generalizalo para cuando \(I\) y \(J\) 
tienen una cantidad finita de generadores.

En general, definimos la potencia de un ideal \(I\) como sigue:
\[
I^0 = 0,\quad I^1 = I, \quad I^n = I^{n-1}I.
\]

Poner ejemplo con las matrices triangulares.

\subsection{Operaciones con ideales}

\paragraph{Suma.}%
Sean \(I\) y \(J\) dos ideales en un anillo \(A\).
La suma 
\[
I+J = \left(I\cup J\right)= \{x+y \mid x\in I, y\in J\}
\]
es también un ideal de \(A\).

\paragraph{Producto.}%
\[
IJ = \left(xy\mid x\in I, y\in J\right)
\]


Será útil recordar la cadena de inclusiones
\[
IJ\ \subset\ I\cap J\ \subset\ I\ \subset\ I+J.
\]

\subsection{Tipos de ideales}

Siempre que tengamos un ideal primo,
debemos pensar en la caracterización con el cociente.
\[
	\mathfrak{p}\text{ es primo } \iff A/\mathfrak{p} \text{ es un campo}
\]
y lo mismo para todo ideal maximal \(\mathfrak{m}\).


Anillos local


\begin{ej}
\begin{enumerate}[label=(i)]
	\item considera
\end{enumerate}
\end{ej}
Anillo noetheriano

El siguiente resultado es muy importante,
pues muchos resultados apoyan su demostración en él.

\begin{thm}[El lema de Nakayama]
Sea \((A,m)\) un anillo local.
Sea \(M\) un \(A\)-módulo finitamente generado.
Si \(M=mM\), entonces \(M=0\).
\end{thm}

\section{Teoría de módulos}

\begin{defi}
Sea 
Un módulo es un grupo abeliano \((M,+)\) dotado de una operación externa
\(\cdot: 	\)
\end{defi}

Esta definición es exactamente la misma que la de \(K\)-espacio vectorial,
excepto que \(K\) es un anillo cualquiera en lugar de un campo.
Por eso, no es un abuso llamar a la teoría de módulos 
``álgebra lineal sobre anillos''.
Sin embargo, 
las consecuencias que tiene permitir a \(K\) ser cualquier anillo
son enormes.
La principal que debemos tener en cuenta es que 
los módulos no tienen que tener una base, mientras
que todo espacio vectorial tiene una.
Esta es la principal distinción.
Este es el hecho que hace a los espacios vectoriales tan agradables.
Por eso, a veces usamos algún método
para pasar de la teoría de módulos a la de
espacios vectoriales, como puede ser 
formar el cociente \(A/m\) para obtener un campo
y así nuestros \(A/m\)-módulos son espacios vectoriales.

\begin{defi}
Módulo generado por un conjunto
\end{defi}

\begin{defi}
modulo f.g.
\end{defi}

\begin{ej}
\begin{enumerate}[label=(i)]
	\item considera
\end{enumerate}
\end{ej}

\section{Secuencias exactas}

Por qué son importantes?
\[
0\to L \to M\to N\to 0
\]

\subsection{Problemas resueltos}

\begin{problema}
Demuestra el tercer teorema de isomorfismo usando el lema de la serpiente.
\end{problema}

\begin{problema}
Da una prueba corta del teorema de intersección de Krull usando el
lemma de ... 
\end{problema}
\end{document}
